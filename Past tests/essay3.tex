\documentclass[12pt,a4paper]{article}
\usepackage[russian]{babel}
\usepackage[utf8]{inputenc}
\usepackage{parskip}
\usepackage{setspace}
\usepackage{wrapfig}
\usepackage{graphicx}
\usepackage{caption}

\oddsidemargin=0 cm
\evensidemargin=0 cm
\textwidth=170 mm
\textheight=260 mm
\topmargin=0 cm
\voffset= -2cm
\pagenumbering{false}
%\newlength{\varheight}
%\setlength{\varheight}{3.1cm}
\setlength{\parindent}{0cm}
\spacing{1.1}
\parskip=2mm
\clubpenalty=10000
\widowpenalty=10000
\captionsetup[figure]{labelformat=empty}

\begin{document}

\textit{``Things do not change; we change.''}

Do we ourselves cause change in our life by making certain decisions, or are we acted upon by events that happen around us?

\vspace{3mm}
\hline
\vspace{5mm}

A person's character is evolving and adapting to the external events through all the life. The brightest highlights in the life change the way of thinking greatly. The same events and entities may seem absolutely different after such a change. On the other hand, the decisions of a person derive his or her future events. I think that a human's character is a complicated system with a strong feedback, which can sometimes influence the environment or just choose a particular one.

First of all, events in the life do change attitude to even the same situations. People often evaluate the possibilities and other events from the point of view of usefulness, how they might be helpful. It is obvious to say that someone's goals change during the life, so what seemed to be (and could really be) useful earlier, became completely useless in a time.

For example, when I was younger, I wanted to get all excellent marks in school. This idea was brought to me by my parents, and seemed to become reality. However, when I entered another school, I started getting ready for more serious events such as olympiads. Very quickly I understood that an excellent GPA became absolutely useless for me, so I reduced my efforts on it.

Secondly, some events, memories or objects may become ``out of date'', it means that the actions associated with those things are no longer considered valid.

An example of a strong outdating influence can be taken from G. Orwell's ``1984''. Before the couple was caught by the mental police, Winston and Julia were in a deep love with each other, which also was created by an external influence. Though, when they both were set free, no shadow of a feeling emerged on their faces. All the memories were pretty alive, but one thing changed --- they became outdated and unapplicable.

However, sometimes a person makes a decision by himself that changes his life. This is en example of the feedback to the environment.

Such situations typically occur when someone has a difficult choice to complete. That can be a choice between jobs, or between a good job and a nice girlfriend. For me, for example, it was a choice of my university. I can clearly say that this decision influenced my life greatly, and it actually influenced others around me.

To conclude, I think that the question of what has been the original source of a change in one's character is difficult. The answers to it may be controversial and diametrally opposite in different occasions.

\textit{32 min}

\end{document} 