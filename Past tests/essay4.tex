\documentclass[12pt,a4paper]{article}
\usepackage[russian]{babel}
\usepackage[utf8]{inputenc}
\usepackage{parskip}
\usepackage{setspace}
\usepackage{wrapfig}
\usepackage{graphicx}
\usepackage{caption}

\oddsidemargin=0 cm
\evensidemargin=0 cm
\textwidth=170 mm
\textheight=260 mm
\topmargin=0 cm
\voffset= -2cm
\pagenumbering{false}
%\newlength{\varheight}
%\setlength{\varheight}{3.1cm}
\setlength{\parindent}{0cm}
\spacing{1.1}
\parskip=2mm
\clubpenalty=10000
\widowpenalty=10000
\captionsetup[figure]{labelformat=empty}

\begin{document}

\textit{``Failure is impossible.''}

Is it really impossible to fail? Are some failures simply unsuccessful attempts to accomplish what we set out to do, or do all failures ultimately provide some benefit, even if we can't see it right away?

\vspace{3mm}
\hline
\vspace{5mm}

Any event in the world may either occur or not occur. The same may be said about the result of an action --- one might either succeed or not. Usually we define the word ``failure'' as a simple opposite to ``success'', such failures happen every day in anyone's life. But every failure may cultivate a desire to try again. So I think that both statements in the assignment are correct.

First of all, no person can live ideally, with success in all parts of his or her life. There's always a thing that one does the first time, without previous experience and often without preparation.

For example, the child tries to stay on his legs, then slowly begins trying to walk. These attempts are essential in his life, as they will bring him the ability to walk. But no one expects that the child suddenly starts, for example, talking like an adult.

However, a failure should not discourage, as it doesn't stop the baby from learning. Every unsuccessful attempt is a sign that you're doing it wrong and you should improve yourself and retry. Failures also provide useful experience, which is incredibly helpful in self-improvement.

Let's say, a student wants to study with a skilled teacher to prepare himself for an exam. Though, the student is not highly advanced in the topic. He finds a talented person which is however unable to explain normally. The teacher is expecting a better fundamental level of knowledge of the student, and fails to explain what he considers obvious but what is new for the student. If the student just stops the clacces and doesn't try again in despair, he'll probably fail the exam. But if he finds another teacher, which is possibly less talented as a user but more as a sharer of his knowledge, then the student will learn the essential material and eventually pass the exam. This example shows us that failures tend to lead, or sometimes even kick a person to success.

To conclude, I think that failure is a rather gentle and abstract termine that just means an unsuccess in a particular task and particular time. Failures should provide new chances to try and great experience, but not discourage and loss of confidence.

\textit{28 min 30 sec}

\end{document} 