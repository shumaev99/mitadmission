\documentclass[12pt,a4paper]{article}
\usepackage[russian]{babel}
\usepackage[utf8]{inputenc}
\usepackage{parskip}
\usepackage{setspace}
\usepackage{wrapfig}
\usepackage{graphicx}
\usepackage{caption}

\oddsidemargin=0 cm
\evensidemargin=0 cm
\textwidth=170 mm
\textheight=260 mm
\topmargin=0 cm
\voffset= -2cm
\pagenumbering{false}
%\newlength{\varheight}
%\setlength{\varheight}{3.1cm}
\setlength{\parindent}{0cm}
\spacing{1.1}
\parskip=2mm
\clubpenalty=10000
\widowpenalty=10000
\captionsetup[figure]{labelformat=empty}

\begin{document}

\textit{``Never do today what you can do tomorrow. Something may occur to make you regret your premature action.''}

Do you agree that it's wisest to put things off if you can?

\vspace{3mm}
\hline
\vspace{5mm}

The problem of time planning has become one of the most important during the last decades. Nowadays people are involved in a vast number of activities which do not require a lot of time but are limited to a strict and a close deadline. I think that the sentence in quote is basically wrong, and an action should be postponed only if an amount of crucial information is missing.

Firstly, if one postpones many activities close to the deadline, he or she will have a rush when this deadline comes.

For example, you have to do a medium-sized homework for Wednesday and a big project for Thursday. If you leave it all for Tuesday, then you'll be out of time doing the homework and preparing for the project simultaneously. To avoid such collapses, you should spend extra resource for monitoring the deadlines of each task. However, even this is not a guarantee of success. For example, on Wednesday you may be given extra homework for Thursday, in addition to the project assignment.

Secondly, if one gets used to leaving the tasks for ``later'', he or she can lose ability to act immediately. The problem is even not in the emergency cases, when you have to do an amount of work in several hours. You just lose the ability to work productively while having pretty much free time. 

That happened to me when I was preparing for my science competition. The point was that I had enough knowledge and an opportunity to take part in the competition twice. So I decided that I had enough time to prepare, and that I also had 2 attempts for the result, and worked not as lot as I should. As I didn't reach the desired goal the first time, I started working more without delaying the tasks for tomorrow. It's not a strange thing to say that this desire has helped a lot.

However, sometimes the tasks have to be left for later. It is an indeed rare situation, but the failure to understand it can also result in a big headache.

One can get in such a situation, for instance, if he or she misses some information to complete the task. The behavior of the person is typically interpreted by others as a ``useless hush'', and a good advice is to read the manuals properly. For example, a girl misses a meeting with her boyfriend due to illness of her mother. The boyfriend may become angry, as she had missed 3 meetings before as well. Though, it wouldn't be a wise decision, obviously.

To conclude, I think that generally it's helpful to solve problems right as you have free time. Complicated and problematic situations may occur when this rule is not the best one. However, a smart brain in your head will probably never be harmful.

\textit{32 min}

\end{document} 
