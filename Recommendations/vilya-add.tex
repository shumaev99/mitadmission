\documentclass[12pt,a4paper]{article}
\usepackage[utf8]{inputenc}
\usepackage{parskip}
\usepackage{setspace}
\usepackage{wrapfig}
\usepackage{graphicx}
\usepackage{caption}

\oddsidemargin=0 cm
\evensidemargin=0 cm
\textwidth=170 mm
\textheight=260 mm
\topmargin=0 cm
\voffset= -2cm
\pagenumbering{false}
%\newlength{\varheight}
%\setlength{\varheight}{3.1cm}
\setlength{\parindent}{0cm}
\spacing{1.1}
\parskip=2mm
\clubpenalty=10000
\widowpenalty=10000

\begin{document} 

I have known Oleksandr Shumaiev since his first National Olympiad, at which he was the absolute winner. I am a jury member at the National Olympiads and Young Physicist Tournaments. I was impressed by his performance in these competitions. We got in touch during the selection exams for the International Olympiad in Physics IPhO 2016. Oleksandr was to solve a set of problems proposed by me, and he coped with this task with flying colors. He showed the best results in all of the exams, finishing with an enormous gap with the runner-up. 

I was the team leader at IPhO 2016, and I viewed Oleksandr's work on the Olympiad examinations. It reflected deep understanding of various areas of physics and scored high gold. After this I became sure that his knowledge covered the 1st year university study program. That's why I invited him to study at Kyiv Taras Shevchenko University and proposed to take the 1st year exams and begin studying the 2nd year program. We communicated a lot in August when he was preparing for these exams, and I gave him advice what subjects and topics should be revised. Eventually Oleksandr passed all exams with excellent scores, and was admitted as 2nd year student at Kyiv Taras Shevchenko University. I can add that I was in charge of one of his exams, namely in Analytic Geometry \& Linear Algebra, and I was pleasantly surprised by his aptitude and deep understanding of the material.

Now Oleksandr Shumaiev is studying in the 5th group of the Quantum Field Theory department under my supervision. I frequently meet with his instructors and receive only positive feedback about his academic achievements. Oleksandr shows great enthusiasm about taking part in scientific research. He decided to begin a research project under my supervision after his TOEFL and SAT exams. Oleksandr already has some experience in this field, as he has done the research of the Duboshinskii pendulum argumental oscillations.

Also, Oleksandr is very communicative, he enjoys sharing his knowledge and taking part in various faculty events. He used to communicate with members of other teams at the International Olympiads and made many new friends abroad. Oleksandr is going to volunteer for the Winter Physics School shortly to teach talented pupils from all over Ukraine. He also wants to participate in the National Physics Olympiad as a jury member. Oleksandr is very responsible about these events and is preparing diligently for them.

I enthusiastically recommend Oleksandr Shumaiev for the MIT Transfer Admission as one of my best students and I'm sure he will definitely succeed in his future studies and hopefully, numerous researches. If you have any questions concerning Oleksandr, please feel free to contact me.

\end{document} 

