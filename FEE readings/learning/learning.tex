\documentclass[12pt,a4paper]{article}
\usepackage[utf8]{inputenc}
\usepackage{parskip}
\usepackage{setspace}

\oddsidemargin=0 cm
\evensidemargin=0 cm
\textwidth=170 mm
\textheight=260 mm
\topmargin=0 cm
\voffset= -2cm
\pagenumbering{false}
%\newlength{\varheight}
%\setlength{\varheight}{3.1cm}
\setlength{\parindent}{0cm}
\spacing{1.1}
\parskip=2mm
\clubpenalty=10000
\widowpenalty=10000

\begin{document}

The Nobel laureate, Carl Wieman, in his interview with Eric Westerwelt, argues that incorporating active learning methods into the curriculum can substantially improve undergraduate education, especially in the Sciences. The brief report on this interview, from NPR Ed, presents some of his core arguments in favor of such a change.

MIT is very interested in improving STEM education, and in exploring active learning strategies. We would like you to carefully consider Wieman�s claims and ideas, test them against your own experiences of learning in the sciences and engineering, in order to offer some insights about how this might be most effective for students.

For the purposes of this essay, write a recommendation, with reasoning and examples, for how active learning should or should not be incorporated into MIT's first-year math and science subjects. This essay should be between 750 and 1250 words, and addressed specifically to MIT's Director of the First-Year Program. For the purposes of this exam, please do not sign your name.

Your recommendation should be supported with a thoughtful analysis and specific examples of classroom practices and their implications for learning. You do not need to present any other research beyond what is referenced in the article and your own or observed experience.

Scorers of your essay will be looking primarily for evidence that you can:

- draw from the available evidence to define the problem with current educational approaches;

- evaluate the position Wieman takes in this interview;

- describe, in concrete terms, a recommendation based on the way you define the problem;

In addition, you should:

- support your thesis with specific detail, solid arguments and convincing evidence, including vivid examples from your own experience;

- cite appropriately when drawing evidence or ideas from the source;

- consider and respond to possible objections to your argument as a means of strengthening it;

- structure the essay logically and coherently, while avoiding a mechanical five-paragraph essay structure;

- write clear, correct, engaging prose.

\newpage

Dear Director of the First-Year Program,

The educational system in most of universities in the world is old-fashioned and no longer effective, so it needs to be revised. The Nobel Prize winner Carl Wieman studied this issue and found out the key problems and the steps required to solve them. Wieman appeals for implementing the so-called active learning techniques instead of big lectures. I am writing you to support his opinion and to convince you to initiate these changes at MIT.

The classical system of education was established in the eighteenth century, when there was a lack of books and not every student could afford them. Professors found a solution: they gave lectures for all of the students simultaneously, in order to use their (professors') time efficiently. Students listened to lectures and wrote down the most important material. So, the ones not having books were not much behind: they could read their notes instead. From then, times have changed, and now everyone can find appropriate books, either in paper or in the internet. However, as the world has become more dynamic, the static and non-interactive format of big lectures has faced new barriers. Carl Wieman explains, why the lectures are not so efficient in teaching students.

Wieman points to a study, in which a series of tests was given to university students a significant time after their lectures. The students showed very low results in those tests. So, the students quickly forgot the information which they acquired during the term, and the very studying appeared to be useless for them. The reason of that is not hard to understand. Humans memorize facts using associations, they associate any new idea with the known ones. When one is required to recall some information, he or she moves in a chain manner through the facts and concepts and finally finds the needed one. In order to establish an association, the brain has to perform a kind of mental processing. But during a lecture, says Wieman, the student doesn't have time to think about all what the professor says, so the information is passing by without effect.

I am a person that devotes a lot of time to studying, and my experience agrees with Wieman's words. In school, my physics teacher used to give me individual lessons. I received much more information on those lessons rather than during ordinary classes with other classmates. To say more, I could hardly understand anything in class, because I wanted the teacher to respond to my thoughts and ideas. In my university I'm facing the same problem, except for nobody is giving me private classes anymore. The only way I can learn something is simply by reading the textbook on my free time. Well, then the lectures are completely useless for me, it is enough to do home assignments and attend midterms and exams. Thinking in this manner, students get disappointed and lose their interest in learning even more.

The technic of active learning is designed to make the studying process more effective. The instructor organizes the class as follows. He sets the problem and describes the basic principles of the solution. After that, students have time to think about the problem and process the ideas of the solution. The instructor answers students' questions and monitors their progress. So, students move towards the correct solution step by step by themselves, and the instructor shows them the direction.

According to Carl Wieman, active learning does not have the drawbacks which make traditional education inefficient. Students participate actively in the learning process, instead of just listening to the stream of data. Each one can organize the information as it is suitable for him, and the instructor fills in what one is missing by answering questions and providing further details. In this way, two goals are achieved simultaneously: students process the ideas and it is easier for them to memorize the concepts, and also they are aided by the instructor in case they do not understand something.

Of course, this method requires a lot of time and concentration from the instructors. Despite this, the overall efficiency appears to be much higher. As I showed earlier, lectures are almost useless for teaching students, so the lecturer simply wastes his time. Students anyway have to come home and revise the entire scope of the lecture using their textbooks. Now we see that active learning not only increases the productivity but also saves both students' and instructors' time.

Active learning is best for math and science disciplines, in which students acquire the main ideas and then solve problems by using these ideas. I would like to recommend the following steps. First, the instructor takes the curriculum and splits it into many subproblems. Next, for each subproblem he finds the problem statement and identifies crucial concepts that are necessary to find the solution. Finally, during classes he focuses the students on the problem and explains the concepts of the solution. 

I believe that MIT is the place where innovations are brought to life, and I would be happy to study in a science class that implements active learning. I'm looking forward to getting your response and reading your remarks on these ideas.

Sincerely yours,

***

\end{document}