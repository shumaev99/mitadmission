\documentclass[12pt,a4paper]{article}
\usepackage[utf8]{inputenc}
\usepackage{parskip}
\usepackage{setspace}

\oddsidemargin=0 cm
\evensidemargin=0 cm
\textwidth=170 mm
\textheight=260 mm
\topmargin=0 cm
\voffset= -2cm
\pagenumbering{false}
%\newlength{\varheight}
%\setlength{\varheight}{3.1cm}
\setlength{\parindent}{0cm}
\spacing{1.1}
\parskip=2mm
\clubpenalty=10000
\widowpenalty=10000

\begin{document}

Write a well-organized, clearly worded essay between 750 and 1250 words that summarizes the essential information presented in the four assigned articles on Blockchain technology. Your task in this assignment is to introduce the ideas presented to your readers and inform them about the central issues, not to persuade them to take a particular action or to espouse a particular viewpoint. Furthermore, you are not being asked to evaluate the relative merits of the ideas expressed in the readings; your approach to the material should be objective, not opinionated.

Synthesize the information in the articles to organize your essay around topics and ideas. Do not just summarize the articles one-by-one. Assume your audience is a group of your peers who have not studied or researched this topic. Do not do any additional reading or research beyond the articles provided, and, for the purposes of writing this essay, do not draw on any previous knowledge you may have on the subject.

You should include, at most, two or three direct quotations. Most of the information from the articles should be reported by paraphrasing (see below) the author�s words. You should document all direct quotations through parenthetical documentation that uses the author�s last name and the page number, if available (e.g., (Iansiti 120)); otherwise, use a shortened form of the title. You also need to cite paraphrased sources by either including a reference to the specific article in the text (e. g. As Iansiti notes...) or a parenthetical citation at the end of the sentence. If you are drawing on information from a source referred to within one of the readings (for instance, when Thompson is quoting Samaras), use the actual speaker in the signal phrase and show that the speaker is quoted in another source through the parenthetical citation (e.g. Kumar claimed... (qtd. in Underwood)). All information derived from the readings must be cited by one of these two methods. Failure to acknowledge the readings as sources of information in your essay will result in a failing score for this essay. You do not, however, need to include a list of Works Cited.

Scorers of your essay will be primarily looking for evidence that you can:

- extract the essential information and ideas from each reading;

- determine how a particular article's information and ideas relate to those in the other articles (for example, whether they support, contradict, or supplement the other material);

- synthesize the ideas and information into a logical and coherent structure of your own, avoiding random lists of points or an article-by-article summary;

- paraphrase authors' language concisely and accurately;

- acknowledge sources of information correctly;

- write clear, correct, engaging prose.

\newpage

The four articles explain the blockchain technology and discuss its benefits, drawbacks and possible applications. Blockchain is a **distributed ledger** (Underwood 1, Iansiti 3) that collects and stores information about financial transactions. The design of the system excludes the possibility to rewrite, modify or delete records. In addition, the records are not kept in a single center, but they are distributed among all of the users of the blockchain network. These facts make blockchain extremely safe and protected against malicious attacks.

In their articles, Rob Marvin and Nick Scott refer to Tapscotts' explanation of the Bitcoin model. Records are kept in blocks which are encrypted with a stable algorithm. Each block is assigned a hash, and every following block includes the hash of the previous one. In this way, blocks are connected in a chain, and hashes can be easily checked to find out whether a block has been tampered with. On the other hand, it is impossible to change a recorded hash value, because it is kept encrypted inside the next block. So, in order to change the history of records, one has to rewrite the whole blockchain. Every 10-20 minutes all transactions in the network are verified and collected, forming a new block. Blocks are shared among a peer-to-peer network, so that there is no particular target for hackers. If one manages to rebuild the 50-gigabyte blockchain for 20 minutes and modifies one, ten or even thousand copies of it in the network, these modifications will not be approved by the millions of users which still hold the correct copy. This gives the blockchain unique stability and reliability.

All four authors agree that blockchain will bring significant changes to business, economics, and finance. Underwood emphasizes the benefits of using a global paying system. Nowadays contracts and transactions are verified and processed by third parties: banks, law firms, governments etc. It takes a lot of time to complete a transaction, and intermediaries charge high percent for their service. Moreover, the payee and the receiver have to share their personal data with the trusted parties, which violates their privacy. In contrast, using blockchain for transactions is fast, secure and cheap: you no more need to disclose your private information, the transactions themselves take not more than 1 second, and no extra fee is charged. Marvin and Underwood show another aspect in which blockchain is useful for developing countries. In countries where there is a high level of fraud or corruption, land registration using blockchain secures its owners from losing the land to fraudulent officials. Blockchains can also be used to provide social security and prove the identity of the legitimate owners.

Iansiti compares the blockchain revolution with the introduction of TCP/IP. Both TCP/IP and blockchain are open-source projects and are maintained by enthusiasts. Iansiti splits the adoption process of TCP/IP into four stages which differ by novelty and complexity, and marks the similar stages of the evolution of blockchain technology. The first stage (single use) represents low-novelty and low-complexity applications such as email. While used in small private networks, TCP/IP gives the opportunity to exchange information directly between computers using email. This is much more convenient than the previous systems of data exchange, which required a connection to be maintained continuously by both parties. Similarly, Bitcoin allows for exchanging assets using blockchain, without resorting to banks and common payment systems. The second stage (localization) comprises technologies with a higher level of novelty but still used by a limited circle of users. Using TCP/IP, corporations designed internal networks in order to exchange and distribute information conveniently. In a similar way, private ledgers can be used to store financial transactions inside companies. On the third stage (substitution) there are low-novelty systems based on the ones from the first stage. However, these are now globalized and require high level of coordination. Examples of such systems based on TCP/IP include online shops and stores. Analogically, digital stores may be selling copyrighted products (Marvin) or bonds inside a blockchain framework. Finally, the fourth stage (transformation) contains innovations that can **change the very nature of economic, social, and political systems** (Iansiti 9).

One of the examples of the fourth-stage innovations is a smart contract. As Marvin explains, smart contract is a set of rules which automates the transaction process. For example, when a customer buys a product from another country, he has to pay and then wait until it will be shipped. This introduces a security breach because the product can be of inappropriate quality or the seller may send a wrong one by mistake. Smart contracts solve this problem. The customer first signs the contract and does not transfer money yet. Only after the product has been shipped and checked for quality, the smart contract triggers the payment.

Blockchains are already used in the fields where there is a need to track the product from the production plant up to the customer. For example, Everledger system keeps records of diamonds in a blockchain, and the customer can verify authenticity of the product (Marvin).

Marvin also shows other possible applications of the blockchain framework. For example, MedRec is a project of a ledger-like medical database which contains patients' medical data, as well as their medical security information. According to Marvin, blockchains can be used to store identity information and serve as ID cards, blocking the possibility of identity theft.

Still, all of the authors agree that the blockchain technology is far from its global adoption. According to Kumar (quoted in Underwood): **Blockchain is not yet ready for prime time**. Underwood and Scott warn about low compatibility between different ledgers. Iansiti reminds that adoption of TCP/IP took a huge amount of time, and development of blockchains is also a matter of far future. The financial system of the world has to undergo substantial changes before blockchain systems can find their places in our lives.

\end{document} 